\section{Introduction} 

Monads (combined with functors and applicatives) form the backbone of functional programming.
While confusing to beginners, the ability to do generalized "iterator style programming" allows for abstractions and better code design. \cite{monads}
They allow for the usage of distinct ideas, for instance, side effects and potential failure, in the same way.

This allows us to write code that is generic and understandable.
Because of this, the natural extension to this was to allow \emph{multiple} of these "wrappers" per value.
This would allow for things such as "output with pure state", where, instead of having \verb|State (IO a)|, where we need to manually deal with both levels of nesting (which breaks do notation), we can instead have \verb|StateT IO a|, which neatly wraps the "inner" monad (\verb|IO| in this case) with state.

This led to the creation of Monad Transformers, Monads that add "capibilites" to other Monads.
Almost ever single monad has a transformer variant, \verb|State : Type -> Type| becomes \verb|StateT : (Type -> Type) -> Type -> Type|, for instance.
Then, when we wish to get the "base" or "regular" version of a monad, we merely pass in the "identity" type, something that just returns the inner type unchanged.

Despite this, there is one very important type that is, at its face, a monad, but not a monad transformer, that being the list monad.
The list monad (and applicative) should conceptually model the idea of a "line of computation".
That is, when the code in \todo{Figure 1} is executed, it takes each possible pair and computes the sum:

\begin{minted}{idris}
	addM : List Nat -> List Nat -> List Nat
	addM xs ys = do 
		x : Nat <- xs
		y : Nat <- ys 
		pure (x + y)
\end{minted}

However, if we wanted to add to this the ability to print out the product of each pair as we were going through, we would be stuck, as we might try one of \todo{figures} 

\begin{minted}{idris}
	addA : List (IO Nat) -> List (IO Nat) -> List (IO Nat)
	addA xs ys = do 
	x : IO Nat <- xs
	y : IO Nat <- ys 
	let z : IO Nat = x * y
	pure (x + y)
\end{minted}

There are two main ways we might make a list transformer, those being below.
\todo{Finish}
