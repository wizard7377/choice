\section{Introduction} 

In both Haskell and Idris, Monads play a very important role in abstraction.
They are central to \verb|do| notation in both languages, which forms the basis of functional abstraction, without which programming with pure "state" would be difficult and unexpressive.

One of the simplest examples, and also one of the first given, is the list monad.
This is because \verb|>>=|, which for a list would have type, \verb|List a -> (a -> List b) -> List b| is exactly the same as an equivalent \verb|flatMap| type \needcite.
Because of this, it is easier to teach relationship between "restricted iterators", and functors, applicatives, monads, traversables, and the like.

Given this utility, and the fact that lists have a very central role

\tikz \graph [binary tree layout]{
	"Root" -> 
	"A" ->[->>] {
		[> ->>]"B",
		"C" -> {"D", "E"}
	}
};