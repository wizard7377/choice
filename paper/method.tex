\section{Free Choice}

The solution to all the above problems was to create a Choice Monad Transformer, closely based off the design of the \verb|logict| package in Haskell and it's \verb|LogicT| \cite{logict}.
However, while this is based off of that package, there are some major differences.
First, however, let us compare the regular List and the \verb|LogicT|

\todo{Cleanup formatting}

\begin{minted}{idris}
	data List : Type -> Type where 
		Nil : List a
		(::) : a -> List a -> List a
\end{minted}

\begin{minted}{idris}
data LogicT : (Type -> Type) -> Type -> Type where 
	MkLogicT : 
		(forall r. 
			(a -> m r -> m r) 
			-> m r 
			-> m r
		) 
		-> LogicT m a
		
unLogicT : 
	LogicT m a
	-> (forall r. 
		(a -> m r -> m r) 
		-> m r 
		-> m r
		) 
		
\end{minted}
\footnote{Ported to Idris}

While these two definitions might look unrelated at first, if we take the seemingly strange step of doing \verb|foldr|, on a \verb|List|, we get something of the form 
\begin{minted}{idris}
	foldr {t = List} : (a -> r -> r) -> r -> List ea -> r	
\end{minted}

and flipping around the order of parameters we get
\begin{minted}{idris}
	foldr {t = List} : List a -> (a -> r -> r) -> r -> r	
\end{minted}

which looks shockingly similar to 
\begin{minted}{idris}
	unLogicT :
		LogicT m a
		-> (forall r. 
		(a -> m r -> m r) 
		-> m r 
		-> m r
		) 
\end{minted}

Indeed, the only difference are the presence of the type \verb|m|.
This works because we can then pass in, if \verb|m = Identity|, \verb|::| and \verb|Nil| to recover a list:
\begin{minted}{idris}
	unLogicT :
	LogicT Identity a
	-> (forall r. 
	(a -> Identity r -> Identity r) 
	-> Identity r 
	-> Identity r
	) 
\end{minted}

Then, simplifying this by $\mathrm{Identity t} \Rightarrow \mathrm{t}$ and \verb|Logic = LogicT Identity|, we get 
\begin{minted}{idris}
	unLogic :
	Logic a -> (forall r. (a -> r -> r) -> r -> r) 
\end{minted}

Which, given universal introduction and elimination (that $\forall x . (A \to B) \rhd (\forall x . A) \to (\forall x . B)$ and that $\forall x . A \rhd A$ given that $x$ does not occur in $A$), is equal to \verb|foldr|.
That is, \verb|LogicT| provides a way to fold together computations.

\subsection{The Cons of Folding}

While this implementation is quite nice, it suffers from a couple problems
\begin{enumerate}
	\item Its use of Continuation Passing Style (CPS) means that its much more complex to compile\needcite 
	\item CPS makes the code harder to understand\needcite 
	\item CPS makes the code harder to debug
	\item It can lead to code that is hard to make monomorphic ($r$ only appears on the left hand side)
	
\end{enumerate}

Because of this, here, we propose \verb|ChoiceT|, a List Monad Transformer that does not use CPS, is fully monomorphic, is intutive to understand, and an extension, rather than proxy, for regular Lists.

\subsection{Choice as a List}

The \verb|ChoiceT| monad transformer is defined in terms of one other type, \verb|MStep|, which is devised as follows:

\begin{minted}{idris}
	
%default total
covering
public export 
data MStep : (Type -> Type) -> Type -> Type where 
	MNil : MStep m a
	MCons : a -> m (MStep m a) -> MStep m a

public export 
data ChoiceT : (Type -> Type) -> Type -> Type where 
	MkChoice : m (MStep m a) -> Choice m a
	
\end{minted}
\footnote{The package itself creates a helper type alias, \textrm{MList}, and has \textrm{MStep} be kind polymorhpic.
	For simplicites sake, these are both changed to improve readability.}
	
In short, \verb|ChoiceT| works by allowing for "delimited continuations in lists".
That is, while in a regular list you can retrive values until you get to the end of the list (which allows you to have arbitrarly long list pattern matches) \verb|MStep| wraps each succesive action in 