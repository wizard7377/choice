\section{Free Choice}

The \href{https://github.com/wizard7377/choice}{choice} library defines 3 different implementations of a list monad, all of which capture "global state".
The first two of these are \verb|LogicT|, based off the \needlink\needcite \verb|LogicT| Haskell package.
the second being \verb|ListT| based off the \needlink\needcite \verb|ListT| Haskell package.

Both of these packages capture the notion of "free" \verb|Foldable|, with \verb|LogicT| doing so through continutaions, and \verb|ListT| doing so with monoidal structure.

However, here we also create a novel \emph{tree} based approach to a list monad. 
Firstly, the definition of it is as follows: 

\begin{minted}{idris}

data TreeT : forall k0, k1. (m : k0 -> k1) -> (a : k0) -> Type where 
	MLeaf : (r : Lazy (List a)) -> TreeT m a
	MBranch : (c0 : (TreeT m a)) -> (c1 : m (TreeT m a)) -> TreeT m a
	
ChoiceT1 : forall k. (Type -> k) -> Type -> k 
ChoiceT1 m a = m (TreeT {k0 = Type} m a)

data ChoiceT : forall k. (m : Type -> k) -> (a : Type) -> Type where 
	MkChoiceT : ChoiceT1 m a -> ChoiceT m a
	
\end{minted}
	
First, \verb|TreeT| defines a binary tree like structure.
Note that this is almost the exact same as the \verb|StepT| from \verb|ListT|, with \verb|StepT| being